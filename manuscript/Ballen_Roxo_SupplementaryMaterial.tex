%%%%%%%%%%%%%%%%%%%%%%%%%%%%%%%%%%%%%%%%%%%%%%%%%%%%%%%%%%%%%%%%%%%%%%%%%%%%%%%%%%%%%%%%%%%%%%%%%%%%%%%%%%%%%%%%%%%%%%%%%%%%%%%%%%%%%%%%%%%%%%%%%%%%%%%%%%%
% This is just an example/guide for you to refer to when producing your supplementary material for your Frontiers article.                                 %
%%%%%%%%%%%%%%%%%%%%%%%%%%%%%%%%%%%%%%%%%%%%%%%%%%%%%%%%%%%%%%%%%%%%%%%%%%%%%%%%%%%%%%%%%%%%%%%%%%%%%%%%%%%%%%%%%%%%%%%%%%%%%%%%%%%%%%%%%%%%%%%%%%%%%%%%%%%

%%% Version 2.5 Generated 2018/06/15 %%%
%%% You will need to have the following packages installed: datetime, fmtcount, etoolbox, fcprefix, which are normally inlcuded in WinEdt. %%%
%%% In http://www.ctan.org/ you can find the packages and how to install them, if necessary. %%%
%%%  NB logo1.jpg is required in the path in order to correctly compile front page header %%%

\documentclass[utf8]{frontiers_suppmat} % for all articles
\usepackage{url,hyperref,lineno,microtype}
\usepackage[onehalfspacing]{setspace}



% Leave a blank line between paragraphs instead of using \\

\begin{document}
\onecolumn
\firstpage{1}

\title[Supplementary Material]{{\helveticaitalic{Supplementary Material}}}


\maketitle


\section{Supplementary Data}

Supplementary Material should be uploaded separately on submission. Please include any supplementary data, figures and/or tables. All supplementary files are deposited to FigShare for permanent storage and receive a DOI.

Supplementary material is not typeset so please ensure that all information is clearly presented, the appropriate caption is included in the file and not in the manuscript, and that the style conforms to the rest of the article. To avoid discrepancies between the published article and the supplementary material, please do not add the title, author list, affiliations or correspondence in the supplementary files.

\subsection{Analytical Pipeline}

All the analyses were run in the cluster of the Museu de Zoologia da Universidade de S\~{a}o Paulo, with different settings of number of nodes and threads as described for each part of the process.

A total of XXXX loci were recovered with the selected scheme. We developed a pipeline for generating posterior tree samples for each locus and then use these distributions, so that topological uncertainty at the locus level is correctly incorporated into the network estimation down the pipeline.

\subsubsection{Bayesian infernce}

We wrote code that automatically generates input files for MrBayes (REF), their respective scripts, a script for submitting jobs using PBS, and then run them in parallel using four independent runs and eight chains (one cold and seven hot), requiring a total of 32 threads per individual MrBayes analysis. All the loci were sorted in four jobs of almost the same size (in cases where the number of loci was not a multiple of 4) and run in the cluster MZUSP01 of the Museu de Zoologia da Universidade de São Paulo (Acknowledge the FAPESP process). The average runtime per locus was 4min for a total of ca. 20h execution for XXX loci.

\begin{verbatim}
# file_preparation.sh
\end{verbatim}

We then summarized the posterior tree distribution per locus combining the results of the four independent runs, applying a burn-in of 50\% for each run. We then converted them to newick using phyx (REF)) and combined them into a single posterior tree file of gene trees

\subsubsection{Concordance Factors}

We used a custom script for generating concordance factors and their relative frequencies for each quartet in the posterior distribution of each gene tree using \texttt{PhyloNetworks} (REF).

\begin{verbatim}
#!/usr/bin/julia
using Distributed
addprocs(15)
@everywhere using PhyloNetworks
#using PhyloNetworks
using PhyloPlots
using CSV
using DataFrames

gtfilename = ARGS[1]

println("Processing $gtfilename...")

# load input trees
trees = readMultiTopology(gtfilename)

# calculate CFs
q,t = countquartetsintrees(trees)
df = writeTableCF(q,t)
cfs = readTableCF(df)

# add dummy columns to df so that it is get-pop-tree.pl-compliant
zeros = repeat([0], nrow(df))

df_zeros = DataFrame(CF12_34_lo=zeros,
                     CF12_34_hi=zeros,
                     CF13_24_lo=zeros,
                     CF13_24_hi=zeros,
                     CF14_23_lo=zeros,
                     CF14_23_hi=zeros)

df_ticr = hcat(select(df, 1:5),
               select(df_zeros, 1:2),
               select(df, 6),
               select(df_zeros, 3:4),
               select(df, 7),
               select(df_zeros, 5:6))
\end{verbatim}

The concordance factor table \texttt{name\_of\_the\_table.csv} that is then used as input for SNAq (REF).

\subsubsection{Initial Tree}

Our ML tree was used as starting tree. Branch lengths were removed as SNAq does not need them for the initial tree.

\begin{verbatim}
code here for calculating the ML tree
\end{verbatim}

\subsubsection{Phylogenetic Network Inference}

Once the initial tree and concordance factor table were estimated, we used SNAq (REF) for estimation of hybrid edges under three different sequential assumptions: \texttt{h = 0}, \texttt{h = 1}, \texttt{h = 2}, and \texttt{h = 3}. SNAq infers the best topologies wich include no, one, two and three hybridisations and score them using the logpseudolikelihood (CHECK THIS). The script below was used for estimating these networks using XX threads in a single core.

\begin{verbatim}
# cite julia script here
\end{verbatim}

\subsubsection{Postprocessing}

Include here any further step such as the test of tree-likeness that we have never run before in the snaq project.

\section{Supplementary Tables and Figures}

For more information on Supplementary Material and for details on the different file types accepted, please see \href{http://home.frontiersin.org/about/author-guidelines#SupplementaryMaterial}{the Supplementary Material section} of the Author Guidelines.

Figures, tables, and images will be published under a Creative Commons CC-BY licence and permission must be obtained for use of copyrighted material from other sources (including re-published/adapted/modified/partial figures and images from the internet). It is the responsibility of the authors to acquire the licenses, to follow any citation instructions requested by third-party rights holders, and cover any supplementary charges.

%% Figures, tables, and images will be published under a Creative Commons CC-BY licence and permission must be obtained for use of copyrighted material from other sources (including re-published/adapted/modified/partial figures and images from the internet). It is the responsibility of the authors to acquire the licenses, to follow any citation instructions requested by third-party rights holders, and cover any supplementary charges.

\subsection{Figures}

%%% There is no need for adding the file termination, as long as you indicate where the file is saved. In the examples below the files (logo1.eps and logos.eps) are in the Frontiers LaTeX folder
%%% If using *.tif files convert them to .jpg or .png
%%%  NB logo1.eps is required in the path in order to correctly compile front page header %%%

\begin{figure}[htbp]
\begin{center}
\includegraphics[width=9cm]{logo1}% This is a *.eps file
\end{center}
\caption{ Enter the caption for your figure here.  Repeat as  necessary for each of your figures}\label{fig:1}
\end{figure}


\begin{figure}[htbp]
\begin{center}
\includegraphics[width=10cm]{logos}
\end{center}
\caption{This is a figure with sub figures, \textbf{(A)} is one logo, \textbf{(B)} is a different logo.}\label{fig:2}
\end{figure}

%%% If you are submitting a figure with subfigures please combine these into one image file with part labels integrated.
%%% If you don't add the figures in the LaTeX files, please upload them when submitting the article.
%%% Frontiers will add the figures at the end of the provisional pdf automatically
%%% The use of LaTeX coding to draw Diagrams/Figures/Structures should be avoided. They should be external callouts including graphics.


\bibliographystyle{frontiersinSCNS_ENG_HUMS} %  for Science, Engineering and Humanities and Social Sciences articles, for Humanities and Social Sciences articles please include page numbers in the in-text citations
%\bibliographystyle{frontiersinHLTH&FPHY} % for Health and Physics articles
\bibliography{references}

\end{document}
