\documentclass[12pt]{article}

\usepackage{natbib} % reference management
%\usepackage{amsmath}
\usepackage[pdftex]{graphicx} % read pdf figures
\usepackage[USenglish]{babel} % document language
\usepackage[utf8]{inputenc} % to input uft-8 characters such as accents from the keyboard
\usepackage[noblocks]{authblk} % for the affil block
\usepackage[colorlinks=true,citecolor=blue,linkcolor=blue,urlcolor=blue]{hyperref} % hyperlinks
\usepackage[margin=1in]{geometry}
%\usepackage[none]{hyphenat} % this package avoids hyphenation. use these two last ones upon submission
%\usepackage{setspace} % set double space between lines

%\usepackage{sectsty}
%\allsectionsfont{\normalfont\normalsize}

%\renewcommand\bibsection{\section*{\centering LITERATURE CITED}}

\graphicspath{{figures/}}

\title{Reticulate evolution as a source of topological conflict and taxonomic complexity in hypostomine catfishes (Siluriformes: Loricariidae))}
\author[1]{Gustavo A. Ballen}
\author[2]{Fabio F. Roxo}
\affil[1]{Queen Mary University of London, gaballench@gmail.com}
\affil[2]{Smithsonian Tropical Research Institute, Panamá, Panamá}
\date{}

\begin{document}

\maketitle

\noindent\textbf{Abstract}---Blablabla.

\noindent\textbf{Keywords:} 

\textbf{RH:} Ballen and Roxo---Reticulate evolution in hypostomine catfishes

\section{Introduction}

\section{Materials and  methods}

\section{Results}

\section{Discussion}

\section{Acknowledgments}

\bibliographystyle{apalike} % citação bibliográfica textual
\bibliography{hypostominae_networks} % in this working directory

\vspace{0.5cm}

\newpage

\subsection{Figures}


\end{document}